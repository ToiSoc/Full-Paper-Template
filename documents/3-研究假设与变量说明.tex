\section{研究假设与变量说明}

\subsection{研究假设}

% \begin{itemize}
%     \item 依据与理由:卷积神经网络(CNN)和长短期记忆网络(LSTM)各自在图像识别和时间序列分析方面展现出强大的性能。CNN擅长捕捉空间特征,而LSTM能够处理时间依赖性。假设一:结合CNN和LSTM的混合模型能够通过深度学习技术更全面地捕捉气象数据的非线性特征和长期依赖性,预期将提供超越传统模型的预测精度。
    
%     \item 依据与理由:时间序列数据的可视化是理解其内在结构和模式的重要手段。CNN-LSTM模型结合了时间序列分析与空间特征提取的能力。假设二:CNN-LSTM模型通过将时间序列数据可视化,能够识别并解析气象数据的复杂结构,从而增强模型的预测能力。
    
%     \item 依据与理由:在大数据时代,处理大规模数据集的效率和准确性是模型成功的关键。LSTM因其独特的记忆机制,适合处理序列数据,而CNN能够高效处理图像数据。假设三:在大数据环境下,LSTM和CNN-LSTM模型将展示出对大规模气象数据集的高效处理能力,确保预测的速度和准确性。
    
%     \item 依据与理由:LGESD模型作为一种细粒度哈希检索框架,已在图像检索任务中显示出高效的性能。假设四:LGESD模型将维持其在大规模图像检索任务中的高效性能,即便面对数据量的显著增加。


% \end{itemize}
\subsubsection{卷积神经网络(CNN)与长短期记忆网络(LSTM)的结合}

卷积神经网络(CNN)在图像识别任务中表现出卓越的性能,主要归功于其强大的空间特征捕捉能力。与此同时,长短期记忆网络(LSTM)在处理时间序列数据方面同样表现出色,这得益于其对时间依赖性的敏感性。基于这两种网络的优势,我们提出以下假设:

\textbf{假设一}: 结合CNN和LSTM的混合模型能够通过深度学习技术更全面地捕捉气象数据的非线性特征和长期依赖性,预期将提供超越传统模型的预测精度。

\subsubsection{时间序列数据的可视化与模型预测能力}

时间序列数据的可视化是理解其内在结构和模式的重要手段。CNN-LSTM模型结合了时间序列分析与空间特征提取的能力,这可能对模型的预测能力产生显著影响。因此,我们提出第二个假设:

\textbf{假设二}: CNN-LSTM模型通过将时间序列数据可视化,能够识别并解析气象数据的复杂结构,从而增强模型的预测能力。

\subsubsection{大规模数据处理与模型效率}

在大数据时代背景下,模型处理大规模数据集的效率和准确性至关重要。LSTM因其独特的记忆机制,适合处理序列数据,而CNN能够高效处理图像数据。这引导我们提出第三个假设:

\textbf{假设三}: 在大数据环境下,LSTM和CNN-LSTM模型将展示出对大规模气象数据集的高效处理能力,确保预测的速度和准确性。

\subsubsection{图像检索任务中的LGESD模型}

LGESD模型作为一种细粒度哈希检索框架,在图像检索任务中已显示出高效的性能。考虑到其在大规模图像检索任务中的潜力,我们提出最后一个假设:

\textbf{假设四}: LGESD模型将维持其在大规模图像检索任务中的高效性能,即便面对数据量的显著增加。

\subsection{变量说明}

在本研究中,所涉及到的符号和参数的相关定义如下表 \ref{tab:parameter_explain} 所示:

\begin{table}[h]
    \caption{符号和参数的相关定义}\label{tab:parameter_explain}
    \centering
    \resizebox{0.9\linewidth}{!}
    {\begin{tabular}{*4{c}}
    \toprule
    符号定义 & 符号说明 & 符号定义 & 符号说明 \\
    \midrule
        $t$       & 时间步索引                       & $I$           & 输入图像             \\
        $h_{t-1}$  & 上一个时间步的隐藏状态            & $T$           & CNN提取的深度激活张量     \\
        $x_t$     & 当前时间步的输入                  & $\widehat{T}$& 全局特征的映射          \\
        $W_f, W_i, W_o$ & 遗忘/输入/输出门权重矩阵     & $\widehat{T}_i^{\prime}$ & 局部特征的映射      \\
        $b_f, b_i, b_o$ & 遗忘/输入/输出门偏置项       & $Q$         & 注意力机制中的查询        \\
        $\sigma$  & Sigmoid激活函数                  & $K$         & 注意力机制中的键         \\
        $\tanh$   & Tanh激活函数                     & $V$         & 注意力机制中的值         \\
        $C_{t-1}$  & 上一个时间步的Cell状态            &   $u$       &        二进制哈希码         \\ 
        $\tilde{C}_t$ & Tanh激活后的候选Cell状态       & $D_{2d}$    & 二维空间衰减矩阵         \\
        $C_t$     & 当前时间步的Cell状态               & $\gamma$    & 空间衰减注意力中的超参数     \\
        $o_t$     & 当前时间步的输出门激活值             & $x_n$       & 第n个元素的空间位置横坐标    \\
        $f_t^*$   & 遗忘门的带权输入                   & $x_m$       & 第m个元素的空间位置横坐标    \\
        $i_t^*$   & 输入门的带权输入                   & $y_n$       & 第n个元素的空间位置纵坐标    \\
        $h_t$     & Cell输出的激活函数                 & $y_m$       & 第m个元素的空间位置纵坐标   \\
    \bottomrule
    \end{tabular}}    
    \vspace{-0.8em}
\end{table}

        % 1    & 当前时间步的输出门激活值     \\
        % 1    & Tanh激活函数         \\
        % 1    & 遗忘门的带权输入         \\
        % 1    & 输出门的带权输入         \\
        % 1    & 遗忘门的激活值          \\
        % 1    & 输出门的激活值          \\
        % 1    & 当前时间步的Cell状态     \\
        % 1    & Cell输出的激活函数      \\
        % 1    & 输入图像             \\
        % 1    & CNN提取的深度激活张量     \\
        % 1    & 全局特征的映射          \\
        % 1    & 局部特征的映射          \\
        % 1    & 注意力机制中的查询        \\
        % 1    & 注意力机制中的键         \\
        % 1    & 注意力机制中的值         \\
        % 1    & 二维空间衰减矩阵         \\
        % 1    & 空间衰减注意力中的超参      \\
        % 1    & 第n个元素的空间位置横坐标    \\
        % 1    & 第m个元素的空间位置横坐标    \\
        % 1    & 第n个元素的空间位置纵坐标    \\
        % 1    & 第m个元素的空间位置纵坐标   \\

