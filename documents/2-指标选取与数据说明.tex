\section{指标选取与数据说明}

\subsection{指标的选取}

在气象预测的研究的中,本文精选了一组关键气象指标,如日均温度、日最高温度、日最低温度、降水量、相对湿度和风速,以全面捕捉气象数据的复杂性。此外,为了增强数据集的深度,我们引入了平均气压这一参数,它为模型提供了额外的信息维度,并在CNN-LSTM模型中证明了其有效性。

在对细粒度哈希图像检索模型进行研究的过程中,我们选定了鸟类、食物类、飞机类以及北美鸟类等多样化的类别作为评估标准。这些精心挑选的指标不仅彰显了本研究的广度和深度,同时也为与现有方法的比较提供了一个公平的基准。借助这些指标,我们得以更精确地评估模型的性能,进而为图像识别领域带来了新的研究视角和分析维度。

\subsection{数据汇总及来源说明}

在本项研究中,我们利用了一套综合性且广泛涵盖的数据资源,确保了该研究在气象学领域以及众多计算机视觉子领域中保持了一定程度的公正性。

\textbf{气象数据}:为确保气象数据分析的可靠性和准确性,我们严谨地采用了国家气象信息中心提供的高标准基础数据。数据集一涵盖了2015年2月至2016年8月这一时段,精心选取了日均温、日最高温、日最低温、相对湿度及风速等关键参数;而数据集二则聚焦于2013年1月至2017年4月,除了包含日均温、相对湿度与风速外,还加入了平均气压指标,以此构建起更为全面多元的气象数据体系。

\textbf{图像数据}:为了进一步丰富研究内容的多样性,并将研究范围扩展到图像识别的尖端领域,我们融合了多个国际公认的公开数据集,经过实验分析之后,我们选取了国际公开数据中的4个不同领域的数据集,具体细节如下所述:
\begin{itemize}[leftmargin=2em]
    \item CUB200-2011数据集:专注于鸟类图像分类,包含200个不同的鸟种,为生物多样性研究提供了丰富的视觉资源;
    \item Food101数据集:集合了101种食物的大量图像,适用于食品识别技术的开发与评估;
    \item Aircraft数据集:该数据集专为飞机识别而设计,收集了全球范围内100种不同型号的飞机图像,在复杂目标识别中具有重要的研究价值;
    \item NABirds数据集:这个数据集包含555个鸟类品种,均来自北美地区,被用于研究物种识别的地域特异性。
\end{itemize}

这种跨域数据集的集成为图像分析算法带来了新挑战,考验模型的泛化能力,在未知的数据上仍能保持高效准确的识别性能,同时也要求模型能够学习到跨数据集通用的特征表示,并且能够捕捉到各个数据集之间共有的以及独特的视觉模式。通过这样的挑战,研究者们被推动去探索更为先进的机器学习技术,包括深度学习、迁移学习、多任务学习、领域自适应等策略,以促进模型对异质数据的适应性和鲁棒性。



