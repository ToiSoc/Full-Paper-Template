
\makeacknowledgement

在本论文开展的研究当中,我们的研究受到指导老师钟老师和曾老师的大力支持和鼓励。在他们的指导下,我们团队完成了论文的定题、数据的挑选、实验的开展、研究方向的拟定和修改等环节,没有他们的帮助,就没有本论文的顺利完成,因此我们对指导我们的老师表示由衷的感谢,同时正是他们的耐心解答了我在研究中遇到的各种问题,我们才能不断积累一小步的成就,不断的增长自身的知识、经验和见解,学会了如何去认真的分析问题,准确把握问题的要点,也非常有效地锻炼了我们的实施研究的能力和应对学术的态度。

我们还要感谢每一位在本论文研究开展期间给予我们教室和实验器材支持的学长学姐们,协助我们去寻找数据、修正数据以及训练模型,尤其是在训练模型的任务中,学长学姐们给予了我们宝贵的意见和建议,让我们以正确而且清晰的思路去编写代码分析数据,并得到合理而且有意义的结果,顺利的完成了大量的数据分析操作。我们团队之间也互相感谢,正是大家的不放弃、不摆烂、互相帮助以及互相鼓励,所有的讨论才能聚焦到一个方向,所有的小任务才高效的完成,一切地分工合作有条不紊,我们团队各成员之间均受益匪浅。

我们也要感谢我们团队成员各自的家人和朋友,在本论文研究实施期间,他们始终在我们前进的道路上不断的鼓励和支持我们。万爱千恩百苦,疼我孰知父母?正是他们的支持,给予了我们很大的动力。使得我们能够在学术研究的大道上奋战到最后一刻,有着一段圆满、极具成就感且毫无后顾之忧的学术旅程。

我们也要感谢本论文研究领域上的先驱者和专家们,正是他们在研究大道的最前面铺路,为我们后来跟上的后辈们提供了大量的成果和数据,给予了我们宝贵的经验参考和领域知识的扩充,使我们得以在研究的大道上稳步前进,从而不断地探索、不断地积累、不断地提升。同时,也感谢那些为我们提供参考文献的个人及图书馆,他们为我们提供了阅读大量参考文献的机会,也正是这些珍贵的、不易获取的参考文献支撑着我们团队往正确的研究方向不断前进,收获属于我们的独特的成果。




\vspace{3em}
\rightline{2024年5月10日}

